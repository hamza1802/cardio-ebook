% Auto-generated LaTeX version of the ebook
\documentclass[12pt,openany]{book}
\usepackage[utf8]{inputenc}
\usepackage[T1]{fontenc}
\usepackage[a4paper,margin=1in]{geometry}
\usepackage{graphicx}
\usepackage{hyperref}
\usepackage{enumitem}
\usepackage{booktabs}
\usepackage{longtable}
\usepackage{parskip}
\setlength{\parindent}{0pt}

\usepackage{emptypage}


\title{Cardio: The Power of Movement \vspace{1cm}\\A Practical Guide to Building Endurance and Heart Health}
\author{CardioGroup \\ ESPRIT Tunisia }
\date{October 2025}




\begin{document}

\frontmatter
\maketitle



\tableofcontents


\mainmatter

\chapter{Introduction}
Cardiovascular exercise, or cardio, involves activities that elevate your heart rate and breathing for sustained periods, engaging multiple muscle groups and promoting overall health. This guide is designed to help readers understand the importance of cardio, offering structured routines, safety advice, and resources to make exercise approachable and effective. Whether you are just starting out or looking to intensify your workouts, this book provides practical strategies to improve heart and lung function, increase stamina, manage stress, and enhance mental well-being.

\section*{Benefits}
\begin{itemize}[left=0pt]
  \item \textbf{Heart \& Lung Health:} Regular cardio strengthens the cardiovascular system, improves circulation, and boosts lung capacity, reducing risks of heart disease and respiratory issues.
  \item \textbf{Energy \& Endurance:} Consistent cardio increases stamina, allowing you to sustain activity longer without fatigue and improving energy levels for everyday life.
  \item \textbf{Mental Wellness:} Engaging in cardio triggers the release of endorphins, helping reduce stress, anxiety, and depression while improving focus and mental clarity.
  \item \textbf{Weight \& Sleep:} Regular exercise supports healthy metabolism, weight management, and enhances the quality of sleep, which is critical for recovery and overall well-being.
\end{itemize}

\begin{quote}
\textbf{Purpose \& Outcomes:} This e-book equips readers with structured cardio plans tailored to different levels, insights into overcoming common barriers, and safety strategies. By following these guidelines, readers can progressively improve fitness, develop a consistent habit, and experience the physical and mental benefits of regular cardiovascular exercise.
\end{quote}

\bigskip
\noindent\includegraphics[width=\textwidth]{indoor-cardio-workout.jpg}

\chapter{Survey Insights}
To better understand cardio habits, a survey was conducted with 50 participants aged 18–40. The questionnaire focused on how often individuals engage in cardiovascular exercise, their motivation levels, and obstacles they face. All responses were anonymous, offering a clear snapshot of current trends and common challenges.

\section*{Key Statistics}
\begin{longtable}{@{}ll@{}}
\toprule
Statistic & Result \\
\midrule
Exercise at least twice per week & 60\% \\
Time constraints as a barrier & 25\% \\
Report low motivation & 10\% \\
Limited access to space or equipment & 5\% \\
\bottomrule
\end{longtable}

\section*{Key Insight}
The survey shows that the majority of people face time constraints and motivational challenges as primary barriers to regular cardio. Many participants mentioned that structured plans, accountability partners, or fitness tracking apps could significantly improve their consistency and adherence to workouts.

\noindent\textbf{Visual:} Survey results (interactive chart on web) — for static PDF include your own chart image and place it in an `images/` folder, then replace the text with \verb|\includegraphics[width=\textwidth]{images/survey-chart.png}|.

\chapter{Workout Plan}
This chapter presents three levels of cardio routines—Beginner, Intermediate, and Advanced. Each plan includes indoor and outdoor options, so you can adapt exercises to your environment, equipment availability, and personal goals. Progression is key: start gradually and increase intensity and duration over time to safely boost endurance and fitness.

\section*{Beginner}
For those new to cardio, start with gentle activities that raise your heart rate without causing excessive strain. Walking for 20 - 30 minutes three times a week builds a strong foundation. Alternating between walking and light jogging improves stamina gradually. Indoor options like low-impact marching or beginner dance workouts provide variety. Light bodyweight circuits—step-ups, squats, and simple core exercises—can complement cardio while building strength.

\section*{Intermediate}
Intermediate routines are designed for those with some cardio experience. Jogging or cycling 30–45 minutes, four to five times per week, enhances endurance. Adding jump rope intervals or 20–40 minutes of dance cardio increases heart rate variability and overall fitness. Interval sessions, alternating between 30–45 seconds of effort and 60–90 seconds of recovery, improve cardiovascular efficiency while maintaining variety to stay engaged.

\section*{Advanced}
Advanced programs are for experienced individuals looking to maximize cardio benefits. High-Intensity Interval Training (HIIT) for 20–25 minutes, sprint repeats, and endurance sessions of 60+ minutes are effective ways to challenge the body. Combining running, cycling, and rowing creates a cross-training approach that reduces repetitive stress and improves overall athletic performance. Always monitor intensity and allow sufficient recovery.

\section*{Sample Weekly Schedule}
A balanced weekly plan includes a mix of cardio intensity, recovery, and mobility work. For example, jogging or cycling on Monday, light recovery walks on Tuesday, HIIT sessions midweek, and endurance activities or sprint intervals later in the week. Active rest days with stretching or yoga help the body recover while maintaining routine consistency.

\noindent\includegraphics[width=\textwidth]{Fall-running-leaves.jpg}

\chapter{Safety Tips}
\section*{Warm-up \& Cool-down}
Warm up 5–10 minutes with dynamic movements; cool down with slow walking and static stretches to reduce soreness.

\section*{Hydration \& Nutrition}
\begin{itemize}
  \item Hydrate before, during (if long session), and after exercise.
  \item Light snack with carbs+protein 1–2 hours before intense sessions.
\end{itemize}

\section*{Common Mistakes}
\begin{itemize}
  \item Skipping warm-up/cool-down.
  \item Rushing progress—increase intensity gradually.
  \item Neglecting rest days.
  \item Poor form—prioritize technique.
\end{itemize}

\section*{Special Considerations}
If you have heart, respiratory or joint conditions consult your doctor. Use Rate of Perceived Exertion (RPE) or heart-rate zones as guidance.

\noindent\includegraphics[width=\textwidth]{beginner-stretches.jpg}

\chapter{Interview Highlights}
\section*{Key Takeaways}
\begin{enumerate}
  \item \textbf{Consistency Over Intensity:} Regular moderate cardio beats occasional intense sessions for long-term results.
  \item \textbf{Find Your Why:} Identify personal motivation—health, stress relief, or social connection—to stay committed.
  \item \textbf{Progressive Overload:} Gradually increase duration or intensity to avoid plateaus and injury.
  \item \textbf{Recovery Matters:} Rest days and proper nutrition are as important as the workout itself.
\end{enumerate}

\noindent\textbf{Video:} Interview with fitness expert — see: \textit{stade run.mp4}

\chapter{Visuals \& Resources}
\section*{Exercise Images \& GIFs}
Examples of image sources (free and high-quality). Download local copies for offline use and include them in an `images/` folder before compiling to embed them in the PDF.

\section*{Recommended Apps \& Tools}
\begin{itemize}
  \item Strava — Outdoor run & ride tracking.
  \item Nike Training Club — Guided training plans.
  \item Tabata Timer — Structure intervals.
\end{itemize}

\section*{Helpful Links}
\begin{itemize}
  \item WHO Physical Activity Guidelines: \url{https://www.who.int/news-room/fact-sheets/detail/physical-activity}
  \item American Heart Association: \url{https://www.heart.org}
\end{itemize}

\chapter{Conclusion}
Cardio is a practical and effective tool for improving physical and mental health. Stay consistent, progress gradually, and prioritize recovery. Small daily improvements add up to meaningful long-term gains.

\begin{quote}
"Your heart is your strongest muscle — train it with care."
\end{quote}

\chapter*{References}
\begin{enumerate}
  \item World Health Organization. (2023). \emph{Physical activity guidelines}. \url{https://www.who.int}
  \item American Heart Association. (2024). \emph{Cardiovascular health resources}. \url{https://www.heart.org}
  \item Smith, J. (2022). \emph{Cardio Training and Endurance Improvement}. (sample citation)
\end{enumerate}

\backmatter
\end{document}
